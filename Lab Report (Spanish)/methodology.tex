\section{Metodología}
Para desarrollar el problema, lo primero que hace el algoritmo es crear un set con n coordenadas enteras aleatorias que se almacenan en una lista. Posteriormente, se toma esta lista y se ordena de forma ascendente según las posiciones en x. Esta lista ordenada es luego utilizada como parámetro de entrada para la subrutina recursiva que se encarga de encontrar el par más cercano. Esta subrutina verifica el tamaño del set de coordenadas, y si es mayor a 3 lo divide en 2 subsets de coordenadas. Esta división se hace de forma recursiva llamandose a sí misma 2 veces, una recibe como parámetro de entrada la primera mitad del set de coordenadas y la otra recibe la segunda mitad. Cuando el set de muestra es de tamaño menor o igual a 3, se aplica el algoritmo de fuerza bruta para hallar el par más cercano de estos. 

Luego, se compara la distancia mínima obtenida de ambos subsets y se queda con el par de coordenadas con la menor distancia mínima entre ambos. Esta distancia es utilizada para comparar puntos de un subset que esten cercanos al otro subset. Si la distancia entre las posiciones de x y y es menor que la distancia mínima, se añaden ambas coordenadas a una lista de candidatos. Esta búsqueda comienza con el par más cercano a la división del set y se va alejando de forma cíclica. El ciclo se detiene en el momento en el que un par de coordenadas excede la distancia mínima. A los candidatos se les aplica también el algoritmo de fuerza bruta y se comparan las distancias mínimas. El programa se devuelve la menor de ellas y las coordenadas asociadas a ella. 


 Para determinar el comportamiento de la gráfica tiempo vs n para cada caso, se ejecuta el proceso anterior contando cada condicional (Si) que utiliza la subrutina recursiva y el tiempo de ejecución de la misma. No se tiene en cuenta los procesos de cración y ordenamineto del set de coordenadas para el análisis. Este procedimiento se realiza 256 veces con un mismo set de coordenadas, y se halla un promedio del tiempo de ejecución y complejidad (uso de condicionales). Luego, los datos obtenidos se guardan en un archivo con 3 columnas: n, comparaciones, tiempo de ejecución. Este proceso se realiza con un set de coordenadas de de tamaño n=1000 que aumenta en factor de 3/2 hasta llegar a n=1000000. Tras la ejecución del algoritmo, se obtiene un archivo con los resultados necesarios para hallar la complejidad del algoritmo y los cuales son usados para graficar los resultados de n vs tiempo con la ayuda de Python y MatLab. Por último, se hace un ajuste de la gráfica para determinar su complejidad y sacar las conclusiones del experimento. 